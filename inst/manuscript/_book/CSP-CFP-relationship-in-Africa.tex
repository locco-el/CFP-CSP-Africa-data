% Options for packages loaded elsewhere
\PassOptionsToPackage{unicode}{hyperref}
\PassOptionsToPackage{hyphens}{url}
%
\documentclass[
]{mitthesis}
\usepackage{lmodern}
\usepackage{amssymb,amsmath}
\usepackage{ifxetex,ifluatex}
\ifnum 0\ifxetex 1\fi\ifluatex 1\fi=0 % if pdftex
  \usepackage[T1]{fontenc}
  \usepackage[utf8]{inputenc}
  \usepackage{textcomp} % provide euro and other symbols
\else % if luatex or xetex
  \usepackage{unicode-math}
  \defaultfontfeatures{Scale=MatchLowercase}
  \defaultfontfeatures[\rmfamily]{Ligatures=TeX,Scale=1}
\fi
% Use upquote if available, for straight quotes in verbatim environments
\IfFileExists{upquote.sty}{\usepackage{upquote}}{}
\IfFileExists{microtype.sty}{% use microtype if available
  \usepackage[]{microtype}
  \UseMicrotypeSet[protrusion]{basicmath} % disable protrusion for tt fonts
}{}
\makeatletter
\@ifundefined{KOMAClassName}{% if non-KOMA class
  \IfFileExists{parskip.sty}{%
    \usepackage{parskip}
  }{% else
    \setlength{\parindent}{0pt}
    \setlength{\parskip}{6pt plus 2pt minus 1pt}}
}{% if KOMA class
  \KOMAoptions{parskip=half}}
\makeatother
\usepackage{xcolor}
\IfFileExists{xurl.sty}{\usepackage{xurl}}{} % add URL line breaks if available
\IfFileExists{bookmark.sty}{\usepackage{bookmark}}{\usepackage{hyperref}}
\hypersetup{
  hidelinks,
  pdfcreator={LaTeX via pandoc}}
\urlstyle{same} % disable monospaced font for URLs
\usepackage[margin=1in]{geometry}
\usepackage{longtable,booktabs}
% Correct order of tables after \paragraph or \subparagraph
\usepackage{etoolbox}
\makeatletter
\patchcmd\longtable{\par}{\if@noskipsec\mbox{}\fi\par}{}{}
\makeatother
% Allow footnotes in longtable head/foot
\IfFileExists{footnotehyper.sty}{\usepackage{footnotehyper}}{\usepackage{footnote}}
\makesavenoteenv{longtable}
\usepackage{graphicx}
\makeatletter
\def\maxwidth{\ifdim\Gin@nat@width>\linewidth\linewidth\else\Gin@nat@width\fi}
\def\maxheight{\ifdim\Gin@nat@height>\textheight\textheight\else\Gin@nat@height\fi}
\makeatother
% Scale images if necessary, so that they will not overflow the page
% margins by default, and it is still possible to overwrite the defaults
% using explicit options in \includegraphics[width, height, ...]{}
\setkeys{Gin}{width=\maxwidth,height=\maxheight,keepaspectratio}
% Set default figure placement to htbp
\makeatletter
\def\fps@figure{htbp}
\makeatother
\setlength{\emergencystretch}{3em} % prevent overfull lines
\providecommand{\tightlist}{%
  \setlength{\itemsep}{0pt}\setlength{\parskip}{0pt}}
\setcounter{secnumdepth}{5}

\author{}
\date{\vspace{-2.5em}}

\begin{document}

{
\setcounter{tocdepth}{2}
\tableofcontents
}
\hypertarget{introduction}{%
\section{INTRODUCTION}\label{introduction}}

\hypertarget{background}{%
\subsection{Background}\label{background}}

Since the publication of `The Great Transformation' (Polanyi, 1957), the concepts of `non-economic' and `nonmarket' successively started to emerge as referring to the internal and external factors that assist markets, firms and other types of institutions and organizations to function efficiently and effectively as well as repair their failures. In the last three decades, studies on the nonmarket field starts to emerge. Among other, the nonmarket field was addressed by Baron (1995), Boddewyn (2003), Hillman (2004), Marcus, Kaufman, and Beam (1987), Preston and Post (1975), Yoffie (1987). However, the definition of nonmarket concept was ambiguous among the various studies. The versatility of the nonmarket concept encouraged Boddewyn (2003) to perform an extensive literature research from various fields of study which provide links to the concept. The roots of nonmarket lie in societal and theoretical choices made over several decades. There exist different perspectives regarding nonmarket, which vary in the way they refer to the internal and external factors that provide order to markets, firms, and other types of institutions and organizations. Four main perspectives provide an adequate overture for the nonmarket concept.

\hypertarget{problem-statement}{%
\subsection{Problem Statement}\label{problem-statement}}

\hypertarget{research-questions}{%
\subsection{Research Questions}\label{research-questions}}

\hypertarget{objectives}{%
\subsection{Objectives}\label{objectives}}

\hypertarget{significance-of-research}{%
\subsection{Significance of Research}\label{significance-of-research}}

\hypertarget{scope-of-research}{%
\subsection{Scope of Research}\label{scope-of-research}}

\hypertarget{chapter-disposition}{%
\subsection{Chapter Disposition}\label{chapter-disposition}}

\hypertarget{literature-review}{%
\section{LITERATURE REVIEW}\label{literature-review}}

\hypertarget{introduction}{%
\subsection{Introduction}\label{introduction}}

This chapter seeks to form the conceptual, theoretical and empirical basis of the research endeavor by critically reviewing extant literature on nonmarket strategy, its antecedents and contingencies. The section begins with an exploration of the meaning and scope of nonmarket strategy. Two key strategies, political and social are identified and discussed. Based on the aim of the work, a more thorough discussion of the social aspects of nonmarket strategy is presented, particularly with regard to its relationship with firm performance. The review culminates in the construction of a conceptual framework that presents the relationships under investigation. The literature review is critical to the study as it forms the foundation for the choice of a methodological approach to the research and a basis for the discussion of results.

\hypertarget{the-non-market-business-environment}{%
\subsubsection{The Non-Market Business Environment}\label{the-non-market-business-environment}}

A review of social-systems and political science literature shows that society can be sub-divided into economic, political, social and cultural subsystems, each with its unique institutions {[}@Voinea2017{]}. Thus, as businesses exist within society, they are not only affected by the economic subsystem (market institutions and organisations) but also political, social and cultural sub-systems (nonmarket institutions) {[}@Baron1995{]}. The concept of the non-market environment has enjoyed scholarly examination in business literature with different thoughts on what it constitutes.

To the neoclassical economist, the non-market environment is exogenous from the organisation and thus consider them as neutral to the firm. In contrast, Organisational theory posits that the non-market environment complements the market environment, preventing or resolving market failure. This school of thought asserts that nonmarket institutions are made up of rules, norms and customs that enable firms to deal with uncertainty, interdependence or transaction failure (Barron, West and Hannan, 1994).

Another perspective is presented by, Political theory which introduces the concept of `voice', the mobilisation of public opinion to compel management to change to avoid market failure (Boddewyn, 2003). In this perspective, the non-market environment is described as the coercive force (petitioning, mobilising opinion, protesting) to cause a change in an undesirable state of affairs.

According to sociology, however, the neoclassical, political and organisational definitions of nonmarket environments are insufficient as they do not consider ``social integration, social institutions, collective interests and motivating values.'' (Voinea and van Kranenburg, 2017). Without these, they believe that organisations cannot survive. Thus nonmarket institutions promote solidarity and social order, ensuring social welfare and protection against market failure (Edwards and Fowley, 1998)

Boddewyn (2003, p.~320) incorporates these various perspectives into a general definition of the nonmarket environment as follows:
``Nonmarket refers to (a) values expressing the purposive pursuit of public interests; (b) internal and external interchange mechanisms of coercion and cooperation that complement and balance competition in a reciprocal manner at various levels of interaction; (c) relationships among market and nonmarket organisations resting principally on their actors' sovereignty rights; and (d) the conflictual integration in the light of their failures of society's economic, political, social, and cultural organisations''.
This definition emphasises the interrelatedness of market and nonmarket aspects of the business environment. Thus as Baron (2006) observes, the long term sustainability of an organisation's competitive advantage is contingent on its efficient and effective management of both the market and non-market environments.

Firms must design elaborate plans of action aimed at managing the institutional or societal context of economic competition. These efforts are referred to as non-market strategy (Lux, Crook and Woehr, 2011). Literature identifies two categories of nonmarket strategy (Mellahi et al., 2016), corporate political strategy (how firms engage with political actors for the benefit of the firm (Hillman, Keim and Schuler, 2004; Lux, Crook and Woehr, 2011; de Villiers, Rinaldi and Unerman, 2014)) and corporate social responsibility (a business organisation's configuration of principles of social responsibility, processes of social responsiveness, and policies, programs and observable outcomes as they relate to the firm's societal relationships (Wood, 1991)). These strategies are discussed in turn.

\hypertarget{corporate-political-activity-cpa}{%
\subsubsection{Corporate political activity (CPA)}\label{corporate-political-activity-cpa}}

Despite the capitalistic nature of the business world, it is not removed from governmental (and by extension political) influence. Due to governments' control over the critical competitive resources of business environments (Wöcke and Moodley, 2015), governments and their policies, form a substantial source of uncertainty for firms (Jacobson, Lenway and Ring, 1993; Engau and Hoffmann, 2011 b)\\
Governments have the authority to make decisions that border on market size (regulations on complementary and substitute products (Scott, 2006), market structure (introduction and elimination of entry and exit barriers) and market demand (taxation that targets consumption) (Fingleton, 2009). Thus a firm's economic or competitive environment and government policy are inextricably intertwined (Baron, 1995). The reach of government influence on businesses has so significantly increased since the 1970's that Weidenbaum (1980) refers to it as a second managerial revolution.

In recognition of this, businesses strive to influence the political systems to their advantage by employing corporate political strategy and action. According to Tian and Deng (2007 p.~341), a firm's corporate political strategy (CPS) is the strategy that they ``\ldots employ to influence the formulation and implementation process of government policy and regulation to create a favourable external environment for their business activities''. Despite its focus on interaction with government, political strategy may extend to a more extensive plexus of stakeholders who influence the firm's actions (Puck, Lawton and Mohr, 2018)

@Hillman2004 identify two significant themes in CPS literature concerning firms CPS behaviour, proactive CPS and reactive CPS. Proactive firms engage actively with the government in attempts to reduce the government's regulation of the firm. This type of engagement is referred to as buffering (Blumentritt, 2003). Buffering implies that a firm is ``trying to insulate itself from external interference or trying to actively influence its environment'' (Meznar and Nigh, 1995, p976). In buffering, firms take actions like petitioning or lobbying lawmakers on the possible impact of legislation and forming trade groups to campaign for or against legislation (Hillman, Keim and Schuler, 2004).

Alternatively, firms can take a reactive posture to government regulation and instead monitor and prepare for prospective regulation so that they are compliant in the shortest possible time when they are passed (Hillman, Keim and Schuler, 2004). This strategy is referred to as bridging behaviour. As Meznar and Nigh (1995, p.~976) assert, bridging occurs when firms ``seek to adapt organisational activities so that they conform to external expectations''.

Hillman and Hitt (1999) examine proactive CPS behaviour further and identify two distinct approaches to it, transactional and relational CPS. Where a firm interacts briefly with government or a political institution on an issue by issue basis, their political strategy is transactional. However, if a firm chooses to maintain a long term association with a government or aligns itself with a political ideology, they are employing a relational, political strategy. Thus, a firm using the relational approach develops their political resources by dealing with government across issues and over time so that whenever a problem arises, they already have the resources needed to address it (Hillman and Hitt, 1999).

Thre is extensive coverage of managerial, political ties in literature; however, research is sparse in developing countries, especially in Africa (Liedong, Aghanya and Rajwani, 2019). Due to the collectivist cultures in these settings, the primary way of influencing governmental policy is by informal and personal ties to political actors (White, Boddewyn and Galang, 2015; Liedong and Frynas, 2018). Consequently, Corporate Political Strategy has received little scholarly interest. Coupled with a lack of data on political donations, investigating corporate political activity as non-market investment activity in developing countries is challenging. Thus, the study focuses on corporate social responsibility as a strategic response to the non-market environment.

\begin{verbatim}
        2.1.2 CORPORATE SOCIAL RESPONSIBILITY (CSR)
\end{verbatim}

Despite enjoying scholarly attention and examination for decades, there is still no universally accepted definition for Corporate Social Responsibility (CSR) (Garriga and Melé, 2004; McWilliams, Siegel and Wright, 2006) and legions of definitions persist in literature (Jones, 1999). Wood (2010, p.~50) describes it as ``controversial, fluid, ambiguous and difficult to research''. As many as 37 different definitions of CSR were identified by Dahlsrud (2008), a figure Carroll and Shabana (2010, p.~3) believe is an underestimation ``because many academically derived definitional constructs were not included''

One of the earliest definitions of CSR was propounded by Friedman (1970) who described the social responsibility of business executives solely conducting business according to their shareholders' desires and making as much money as possible within the basic rules of society, legally and by ethical custom. This idea is referred to as the neo-classical view of CSR (Moir, 2001). Neo-classical scholars believe that any investment of effort, financial resources or time in anything apart from increasing shareholder's wealth is unacceptable because the only true stakeholders of the business are the owners who entrusted their resources to management.

Contemporary schools of thought, (the stakeholder theory) however, acknowledge the fact that a firm's responsibility is not only to its ``core'' or ``traditional'' stakeholders (the shareholder) but to a myriad of stakeholders who interact with the entity on various levels. Prominent amongst the proponents of the stakeholder view is Freeman (1984), who describes the firm as a series of connections of stakeholders that the managers of the firm attempt to manage. According to him, a stakeholder is ``any group or individual who can affect or is affected by the achievement of the organisation's objectives'' (Freeman, 1984, p.~46). These stakeholders, by their connection to the firm, provide a social sanction to firms and so requires that the firms, in return, contribute to their growth and development (Devinney, 2009).

The stakeholder perspective pervades recent literature on CSR. For instance, Strike, Gao and Bansal (2006, p.~852) emphasis the role of the stakeholder in defining CSR when they state that ``CSR is the set of corporate actions that positively affect an identifiable social stakeholder's interests and does not violate the legitimate claims of another identifiable social stakeholder in the long run''. In this definition, an emphasis is placed on CSR being a stakeholder specific construct. The stakeholder view seeks to promote the delivery of benefits by aligning and connecting with stakeholders to increase satisfaction and identification with the firm's brand (Luo and Bhattacharya, 2006).

As Wood, (1991, p.~693) states, ``CSR is a business organisation's configuration of principles of social responsibility, processes of social responsiveness, and policies, programs and observable outcomes as they relate to the firm's societal relationships''. Carroll, (2008) provides an apt summary of the concept by explaining that CSR is the commitment of business to contribute to sustainable economic development by working with employees, their families, the local community and society at large to improve their quality of life.

Another perspective of CSR sees it as an extension of corporate governance. Here, management has responsibilities to the ``\ldots{} fulfilment of their fiduciary duties towards the owners, to fulfilment of analogous fiduciary duties towards all the firm's stakeholders'' (Sacconi, 2004, p.~6). Thus the meaning of `governance' transcends the allocation of property rights and definition control of over management. Similar to the perspective of the neo-institutional theory, this perspective of CSR emphasises the fact that management does not owe a fiduciary duty to only shareholders (the mono-stakeholder view of neo-classical theory) but every connected party of the business (the multi-stakeholder view). The claim of the existence of a fiduciary duty takes away the perceived voluntary nature of CSR proposed by some schools of thought. However, as each stakeholder is different and interacts with the firm in its unique way, identification and understanding of the link between the firm and each stakeholder is crucial to gauge what fiduciary duty is owed them.

In summary, CSR is the idea that the management of firms are not only accountable to the shareholders whose resources they are custodians of, but also every person, entity or institution who interacts with, affects and is affected by the business. Therefore, management must consider the wellbeing of these various stakeholders.

\begin{verbatim}
            2.1.2.1 HISTORY AND DEVELOPMENT OF CSR
\end{verbatim}

Whiles the modern terminology ``CSR'' did not emerge till after the Great Depression of the 1930s and World War II in the 1940s, in America, manifestations of altruism, charity, and social give-back, ethical codes, community service, and corporate managers as public trustees have existed in business even before the 1920s (Carroll, 2008).

It was Abrams (1951) who began the conversation on CSR by proposing that corporate executives (referred to as businessmen in that era) must conduct their businesses in the best interest of not just the shareholders but the myriad of stakeholders that interact with the firm. However, Bowen (1953) is the one credited with bringing CSR to the fore of scholarly interrogation through his seminal work `Social Responsibilities of the Businessman'. As such, he is considered the `modern father of Corporate Social Responsibility' for his groundbreaking exposition on social responsiveness, stewardship, social audits, citizenship and stakeholder theory (Boatright, 2001), which many found to be ahead of its time (Carroll, 2008; Wood, 2010). He provided one of the earliest definitions of CSR as ``the obligations of businessmen to pursue those policies, to make those decisions, or to follow those lines of action which are desirable in terms of the objectives and values of our society'' (Bowen 1953, p.~6).

Almost a decade later, Bowen's concept was adopted by Frederick (1960, p.~60), who recognised that the social responsibility of the businessman is to ``oversee the operation of an economic system that fulfils the expectations of the public''. That, in turn, means that the economy's means of production should be employed in such a way that production and distribution should enhance total socio-economic welfare.

According to Frederick (1960), the evolution of CSR took place in four phases beginning from the 1950s (1950's to '60s, 1960s to '70s 1980's to '90s and 1990s to 2000s). He describes the 1950's to '60s as a time of corporate social stewardship. According to him, three main themes were prevalent in that day with regards to social responsibility: ``corporate managers as public trustees and stewards of broad-scale economic interests; an executive duty to balance the competing claims of employees, customers, owners and the public; and philanthropic support of the worthy social causes'' (Frederick, 1960, p.~524). In other words, the focus was on how businesses could be of benefit to society.

However, in 1958, Theodore Levitt raised concerns about the dangers of pursuing ambiguous corporate objectives such as CSR-related activities. Following from this, Friedman (1970), also pointed out that the sole stakeholder of a business and whose welfare should be sort is the shareholder. Thus a firm' management must focus all its time, resources and energy to increase shareholder wealth. These arguments sparked debates about how CSR affects the financial performance of firms, whether engaging in CSR does not distract managers from profit-making tasks, and the likelihood that funds could be misappropriated through CSR activities (Ashrafi et al., 2020). Consequently, in the 1970s, these debates fueled a proliferation of studies that examined the effect of CSR on financial performance, especially in the long run.

In the 1970s CSR concepts and practices burgeoned and became formalised in business practice (Carroll, 2008). Frederick (1960) believes that the CSR agenda was spurred on by a rise in social demands for businesses to be more responsible. America at the time was plagued by war, poverty, urban decay, racism, sexism and pollution, however, the government could not do it all; therefore attention was turned to big firms to respond to the prevailing situation (Wood, 1991). Thus Frederick (1960) refers to this period as the era of corporate social responsiveness marked by conceptualisations and investigations of how companies responded to social demands. Muirhead (1999, p.~15) notes that the focus on philanthropy continued through till the mid -1980s, describing it as the period of ``growth and expansion'' of corporate contributions. However, the definition of socially responsible behaviour expanded to include, provision of better work conditions and better customer care (Heald, 1970).

Davis (1973 p.321), described CSR as ``the firm's obligation to evaluate in its decision-making process the effects of its decisions on the external social system in a manner that will accomplish social benefits along with the traditional economic gains which the firm seeks''. By this, he argued it was possible to balance CSR with the firm's responsibility to increase shareholder wealth and that indeed it was in the long term interest of the firm to do so. Baumol (1970) also propagated the widely accepted notion of `` enlightened self-interest' which encouraged businessmen to engage in CSR not just because they ought to but because they recognise the benefits of CSR to the firm especially in the long run (Lee, 2008).

Another significant development in CSR literature was the conceptualisation of Corporate Social Performance (CSP) by Ackerman (1973) as an extension of CSR. Sethi (1975) further developed a structural framework to facilitate the analysis of CSR and its linkage to CSP. Finally, Carroll, (1979, 1991) into what is now known and widely accepted as Carrolls CSR pyramid. In (1979), Carroll developed a model that integrates Corporate Social Responsibility, Corporate Responsiveness and Social Issues. In doing so, he defined CSR as consisting of ``the economic, legal, ethical, and discretionary expectations that society has of organisations at a given point in time'' (Carroll, 1979, p.~500). Specifically, the firm is expected to make a profit whiles abiding by the laws established by the society's legal system. However, the ethical and discretionary/philanthropic components were presented as ``going beyond the minimum'' required of the firm to be beneficial to society.

The work by Carroll (1979) laid an important foundation for the discourse on CSR and CSP. Noteworthy extensions of Carroll's work include the redefinition of his three components (CSR, social responsiveness, and social issues) by Wartick and Cochran (1985) into a framework of principles, processes, and policies. This framework then served as the material for Wood's (1991) multi-level framework for corporate social behaviour. Wood's framework relates CSR principles at the institutional, organisational, and individual levels to processes of responsiveness including environmental assessment, stakeholder management, and issues management; and `policies' developed by corporations to address social problems.

The 1980s and 1990s, the era of moving beyond activism, (according to Frederick (1960)), saw a shift from an ethics orientation to a performance orientation, and from a macro level to a micro level (Carroll and Shabana, 2010). Theories such as Stakeholder Theory, Resource-Based Theory, and Institutional Theory, were propounded to explain the place of CSR in the corporate arena further. It was during this time that Freeman (1984) presented his stakeholder approach to strategic management that explained that the survival of a corporation is affected not only by shareholders but also various other stakeholders such as employees, governments and customers. Thus he defined stakeholders as ``any group or individual who can affect or is affected by the achievement of the organisation's objectives'', (Freeman 1984 p.~46). His thesis served as the basis for the work of Donaldson and Preston (2007, p.~34) who identified three perspectives of the stakeholder theory: descriptive (how corporations behave), normative (how corporations should behave), and instrumental (how behaviour affects corporate performance). The stakeholder theory is vital to the CSR discourse because by defining who the stakeholders of a firm are, it provides a basis for understanding how to engage with them effectively to create value for the organisation.

Another noteworthy CSR related theory that burgeoned during the 1980s and 1990s is the Resource-Based Theory. Scholars such as Rumelt (1984) and Wernerfelt (1984) argued that a firm could achieve competitive advantage through the deployment of specific corporate resources. Their thesis was further refined by (Barney, 1991, p.~116) who suggested that ``sources of sustained competitive advantage are firm resources that are valuable, rare, imperfectly imitable, and non-substitutable'' This theory is vital to the CSR debate because many scholars view CSR as a strategic resource of firms. Litz (1996), for instance, found that developing social and ethical competencies have the potential of strengthening corporate capabilities that create a competitive advantage.

The institutional theory, propounded by Selznick (1948) holds the view that firms conform to the socially accepted norms in a given business environment in recognition of the fact that they need society's approval to survive. In the words of Meyer and Rowan (1977, p.~340), Institutional Theory proposes the idea that ``organisations are driven to incorporate the practices and procedures defined by prevailing rationalised concepts of organisational work and institutionalised in society. Organisations that do so increase their legitimacy and their survival prospects, independent of the immediate efficacy of the acquired practices and procedures'' . Firms who conform to the norms of the social context within which they exist enjoy a good reputation and legitimacy that leads to more stability through the creation of social acceptance and loyalty, ability to attract highly qualified personnel, and the acceptance of professional bodies (Oliver, 1991). Institutional Theory paved the way for the examination of the CSR phenomena in a context-specific manner across many geographical jurisdictions (Beliveau, Cottrill and O'Neill, 1994; Campbell, 2007; Matten and Moon, 2008)

The late 1990s saw the expansion of the scope of CSR to include environmental aspects of corporate activity. With the publication of the Brundtland Commission Report (WCED, 1987), which introduced the idea of sustainable development, the environmental impact of organisations' was thrust into the limelight and became a salient issue. Then came the proliferation of the triple bottom line concept which emphasised `the simultaneous pursuit of economic prosperity, environmental quality and social equity' (Elkington, 1997). Thus CSR took on a new perspective. The European Commission (2002, p.~1) for instance defined CSR as being
``\ldots about companies having responsibilities and taking actions beyond their legal obligations and economic/business aims. These wider responsibilities cover a range of areas but are frequently summed up as social and environmental where social means society broadly defined, rather than simply social policy issues. This can be summed up as the `triple bottom line approach: i.e., economic, social and environmental.'

Though CSR has a long history, it has recently blossomed as an idea, especially in literature with over 40\% of CSR articles have been published after 2005 (Aguinis and Glavas, 2012). (Ashrafi et al., 2020) describe the current theme of CSR as encapsulating the ideas of stakeholder management, the triple bottom line of economic, social, and environmental performance (Aguinis and Glavas, 2012) and the creation of shared value (i.e., shareholders' value and stakeholders' value) has also become an integral part of the contemporary CSR (Bansal and DesJardine, 2014; Carroll, 2015). However, Landrum, (2018) suggests that as the CSR conversation progresses, competing and complementary themes such as business ethics, corporate citizenship, and corporate sustainability, will receive increasing attention in literature and practice.

\begin{verbatim}
            2.1.2.2 CORPORATE SOCIAL PERFORMANCE
\end{verbatim}

As the study of Corporate Social Responsibility has grown, many related terminologies have emerged to express various aspects of the subject (Lu et al., 2014). Corporate Social Performance (CSP) is one such term linked to the measurement of CSR. Wood (1991) defines CSP as ``a business organisation's configuration of principles of social responsibility, process of social responsiveness, and policies, programs, and observable outcomes as they relate to the firm's societal relationships''.

Carroll (1991), distinguishes CSP from CSR by stating that Corporate Social Responsibility connotes obligation and accountability to society, Corporate Social Responsiveness connotes action and activity, while Corporate Social Performance connotes outcomes and results. Likewise, Marom (2006) (2006) suggested that CSP and application of CSR applicable and a way of putting it into practice. CSR cannot be measured since it is not a variable but CSP can be operationalised by the use of measurable proxy variables (though challenging to measure) (Van Beurden and Gössling, 2008). Visser (2010) pointed out that for practical purpose, CSP might be seen as an extension of the concept of CSR, one that focuses on actual results achieved rather than the general nominal notion of businesses' accountability or responsibility to society

Lu et al.~(2014) observes that the two constructs have used interchangeably in literature to refer to measurable outcomes of CSR activities. For our purposes, we follow Lu et al.~(2014) by maintaining the terminilogy used by individual studies as we discuss their exposition on the relationship between CSR/CFP and CFP
2.1.2.2.1 MEASURING CSR/CSP
Constructing a universal measure for CSR/CSP has been fraught with challenges (Galant and Cadez, 2017), chief amongst them is the lack of consensus on the theoretical meaning of the CSR concept (Dahlsrud, 2008). Further, the idea is multidimensional with relatively heterogeneous dimensions (Carroll, 1979). From literature, CSR is measured as either one-dimensional or multi-dimensional (reputational indices, content analysis and surveys) constructs.

One-dimensional constructs focus only on a single dimension of CSR, for example, environmental management or philanthropy. The environmental aspects of CSR have been measured using data on firms investment in pollution control (Tang, Fu and Yang, 2019), extent of carbon-reduction strategy employed (Cadez and Czerny, 2016), use of eco-control (Henri and Journeault, 2010), proportion of toxic waste recycled (Al-tuwaijri, Christensen and Ii, 2004) and compliance with global environmental standard (Dowell, Hart and Yeung, 2000) etc.

In the case of philanthropy, constructs have been based on, amount donated to charitable causes (Lin, Yang and Liou, 2009), increase in philanthropic donations (Carnahan, Agarwal and Campbell, 2010) and public health policies (Naranjo-Gil, Sánchez-Expósito and Gómez-Ruiz, 2016).

One dimensional constructs of CSR offer the advantage of accessible data. However, they fail to adequately measure the CSR of the firm as they don't present a holistic picture. For instance, if a firm focuses on environmental issues to the neglect of customers, a one-dimensional construct based on environmental performance will rank them high in CSR but a multi-dimensional construct will incorporate the low customer performance (Galant and Cadez, 2017).

A perusal of literature shows the use of reputational indices and rankings such as the MSC KLD 400 social index and the Fortune magazine reputation index. Despite having the advantage of providing ready to use information, (thus minimising data collection effort) and allowing for comparison across firms, these indices also have many weaknesses (Galant and Cadez, 2017)

Graafland, Eijffinger and Smidjohan (2004) point to the fact that these indices are typically compiled by private firms that have their own agendas and do not necessarily use scientific methods. Further, rating agencies often merely provide aggregated CSR scores even though researchers may sometimes be only interested in specific CSR dimensions. The second major weakness is the rating agencies' limited coverage of firms. In terms of geographic area, many indices cover a particular region or country especially developed economic settings. Coverage is also limited in terms of the number of firms rated. Typically, indices concentrate on large and publicly listed companies. Some reputation indices like the MSCI KLD index and the Dow Jones Sustainability index exclude companies operating in industries considered non-sustainable like tobacco, firearms, alcohol, adult entertainment, etc. In effect, many socially and environmentally responsible companies may not make it onto the list due to their size, geographic location or industry affiliation (Adam and Shavit, 2008).

Researchers also employ content analysis of corporate communication to measure CSR performance. Content analysis generally entails determining the constructs of interest, seeking information about these constructs and codifying qualitative information to derive quantitative scales that can be used in subsequent statistical analyses. Content analyses differ concerning the number of dimensions appraised and coding sophistication. A relatively simple way of coding is counting words or sentences (Aras, Aybars and Kutlu, 2010) in reports and publications on the specific CSR issue under consideration (e.g., CO2 reduction) and assigning binary variables (`0' and `1') if a particular topic is mentioned. If several dimensions of CSR are being appraised, a binary score can be assigned to each dimension and then an integrated score can be determined (Abbott and Monsen, 1979)

A more advanced way of coding is pre-specification of CSR dimensions of interest and assigning interval scores, similar to Likert scales, to each CSR issue under consideration. One of the earliest attempts of pre-specification of dimensions is the Social Involvement Disclosure scale (SID) by Abbott and Monsen (1979). Their appraisal included 24 CSR indicators grouped in six categories (environment, equal opportunity, personnel, community involvement, products and other). In a more recent study, Yang, Lin, and Chang (2009) rated companies over five different CSR dimensions (employee relations, environment, shareholder relations, product quality and relations with providers and customers, community) on a 0--5 rating scale (where 0 = fulfilment of no criteria and 5 = fulfilment of all criteria).

Karagiorgos (2010) and Chen, Feldmann, and Tang (2015) based their content analysis on GRI reports. More specifically, Karagiorgos (2010) used 26 indicators derived from GRI reports which were divided into two groups (social performance indicators and environment performance indicator) and rated on a scale from 0--3 (0 if the indicator is not taken into account, 3 if indicator is fully taken into account). Similarly, Chen et al.~(2015) used the 45 GRI indicators. Each indicator was scored on a 1--5 scale (1 = indicator not reported; 5 = indicator fully reported) by multiple raters.

The key advantage of this method is flexibility for the researcher. A researcher can specify CSR dimensions of interest, collect data according to those dimensions and code data numerically for further use in statistical analyses.The main weaknesses of this approach is the researcher subjectivity embedded in all stages of the research process from the selection of CSR dimensions of interest, collection of data, interpretation of data and coding of data. Another important drawback is reporting bias. CSR reporting is largely voluntary hence many organisations fail to report on their CSR activities even if they do engage in them. Such activities are obviously likely to go undetected by the researcher. Even if the companies do disclose CSR-related data, such data needs to be interpreted carefully as companies often immerse themselves in impressions management to create a more favourable image of their company through biased reporting (Turker, 2009).This is difficult to detect unless the researcher is knowledgeable about the firms' socially responsible actions or if the report has been externally audited.

A questionnaire-based survey is typically used when a particular company is not rated by a rating agency and corporate reports are unavailable or insufficient for meaningful content analysis. In such cases, researchers need to collect primary data about CSR by sending questionnaires to knowledgeable respondents or interviewing them. One of the earliest questionnaire surveys concerned with CSR was conducted by Aupperle, Carroll, and Hatfield (1985). The measurement instrument was based on Carroll's (1979) four components of CSR (economic, legal, ethical and discretionary) and included 80 items, organised in 20 sets of statements (each set contained four statements; one for each component of CSR). Respondents were asked to allocate up to 10 points to each set of statements on CSR.

For purposes of studying the CSR--CFP link, Rettab, Brik, and Mellahi (2009) combined different constructs for collecting data on CSR and CFP using a questionnaire. In a more recent study, Gallardo-Vázquez and Sanchez-Hernandez (2014) developed a CSR measurement scale intended to appraise social, economic and environmental dimension of CSR. This method's main advantage is similar to that of content analysis. It provides great flexibility for the researcher in terms of specifying the dimensions of interest and collecting data about these dimensions. The likely drawback of this method, in addition to general limitations of survey research, is response bias. The bias occurs at two levels. Selection bias will likely arise as more socially responsible firms are more likely to respond than firms that are less socially responsible (Cadez and Czerny, 2016). Attitude bias is to be expected as respondents may provide socially desirable answers even though their actual behaviour may differ (Epstein and Rejc-Buhovac, 2014). An alternative for overcoming this drawback may be to collect data not only from firms, but also (or solely) from their stakeholders.

\begin{verbatim}
            2.1.2.3 THE BUSINESS CASE FOR CSR
\end{verbatim}

Berger et al.~(2007) outlined three ways in which organisations' integrate CSR into their core business functions. According to them, a firm can either take a social values-led approach, a business-case approach or a syncretic stewardship approach.

A social value-led approach considers CSR an integral part of the culture and strategy of the organisation without recourse to any particular economic motivation. This approach is akin to what Vogel (2005) describes as the `old style' CSR of the 1960s and 1970s which was motivated by social considerations. ``While there was substantial peer pressure among corporations to become more philanthropic, no one claimed that such firms were likely to be more profitable than their less generous competitors'' (Vogel, 2005, p.~20)

The business-case model and the syncretic stewardship model, on the other hand, are motivated by financial gain. The distinction, however, is that in the business case model, a clear link between CSR and financial performance must exist for CSR to be considered worthwhile (Berger et al.~2007). Thus Carroll and Shabana (2010) refer to it as the narrow view of CSR. However, the syncretic approach accepts CSR as `management philosophy, an overarching approach to business' (Berger et al.~2007, p.~144). The syncretic approach takes a broad view on CSR (Carroll and Shabana, 2010) and thus values and appreciate the complex relationship between CSR and firm performance. In the broad view, the business is seen as dependent on society and vice versa. Thus any investment in CSR is seen as beneficial to the business even if there are no immediate and tangible gains (Porter and Kramer, 2006).

In the recent past, the business model of CSR has gained prominence as CSR theories have shifted from an ethics orientation to a performance orientation (Lee, 2008) aided by a shift from a macro level of analysis to the organisational level where an examination of the effects of CSR on firm financial performance is possible (Carroll and Shabana, 2010). In other words, the proponents of CSR no longer promote it as the ``right thing to do'' but instead as a means of ``doing well'' in business (Vogel, 2005) or the business case for CSR (Schreck, 2011).

As Berman, Wicks, Kotha, and Jones, (1999) explain, a firm's stakeholders must be managed to assure revenues, profits, and, ultimately, returns to shareholders. This view is referred to as the instrumental stakeholder theory. (Jones et al.~2018). Since the community within which the firm operates controls critical resources it needs to survive (e.g.~raw materials and brand loyalty), the firm will do well to manage its's interactions with the community (Salancik and Pfeffer, 1978).

Research has shown that firms with positive brand image enjoy benefits such as the ability to attract and retain highly qualified job seekers (Backhaus, Stone, and Heiner, 2002; Greening and Turban, 2000), commitment of staff (Dutton, Dukerich, and Harquail, 1994), customer loyalty (Bhattacharya and Sen, 2003) and better market performance (Barnett and Salomon, 2006; Graves and Waddock, 1994; Johnson and Greening, 1999). Further, drawing from the Resource-Based View, CSR is seen as a means of differentiation which results in the creation of an intangible assets such as innovation (Klassen and Whybark, 1999), human resources (Russo and Harrison, 2005), and organisational culture (Howard-Grenville, Hoffman, and Wirtenberg, 2003), improved efficiency and ability to use the firms assets most competitively (Surroca et al., 2010).

Additionally, corporate reputation theory suggests that a good reputation acts as insurance against loss of firm value in times of crises (Godfrey et al., 2009; Schnietz and Epstein, 2005). It posits that stakeholders discriminate in their response to adverse events based on their perception of the intentions of the perpetrator. Therefore, a firm with a positive track record is likely to face less severe sanction from stakeholders than one with little or no CSR based reputation (Godfrey et al., 2009). In other words, the CSR reputation of a firm can provide a buffer that absorbed the impact of adverse events.

\begin{verbatim}
    2.2 DETERMINANTS OF NON MARKET INVESTMENTS
\end{verbatim}

According to Aguinis and Glavas (2012), CSR is determined by institutional, organisational and individual factors. Institutional factors refer to standards or certifications (Christmann and Taylor 2006), as well as the socio-cultural context of the country under study (Brammer et al., 2009); The organisational, factors include variables like firm size (Waddock and Graves, 1997), profitability (De Villiers et al.~2011) or corporate structure and governance (Gamerschlag et al., 2011; Johnson and Greening, 1999); and the individual factors, are CEOs' or managers' values with the emphasis on stakeholders' interests (Agle et al., 1999) or employees' values and individual concern for CSR issues (Mudrack, 2007)

\begin{verbatim}
        2.2.1 SOCIO-CULTURAL DETERMINANTS
\end{verbatim}

Hofstede (1980) emphasised the role of the socio-cultural values of countries in determining managerial decision making and thus the behaviour of companies. The country-specific institutional arrangements in politics, law, economics, and the family can have a profound effect on the definition of pertinent roles such as owner, manager, employee, consumer, and citizen and thus determine the scope of CSR expression (Jones, 1999) Ortas et al., 2015). Particular national and institutional arrangements can either actively promote social responsibility, latently sustain it, or actively discourage it (Jones, 1999).

Institutional theory has long established that organizations are embedded within broader social structures, comprising different types of institutions that exert significant influence on the corporations' decision-making (e.g., Campbell, 2007; Campbell, Hollingsworth, and Lindberg, 1991). Moreover, recent work in CSR has argued that CSR activities are framed vis-a`-vis the social context, and are thus influenced by the prevailing institutions in such settings (Jackson and Apostolakou, 2010).

Empirical evidence shows that the political, education, labour, financial, and cultural systems of a country have a significant effect on CSP variations with the political system being the most important and the financial system the least important in terms of estimated economic impact (Campbell, 2007; Ioannou, 2012). Therefore, because the current study is set in the developing world context, the next section explores the unique context of developing countries with respect to CSR.

\hypertarget{corporate-social-responsibility-in-developing-countries}{%
\paragraph{CORPORATE SOCIAL RESPONSIBILITY IN DEVELOPING COUNTRIES}\label{corporate-social-responsibility-in-developing-countries}}

A perusal of literature on the development of CSR (for instance, Frederick, 2006, Campbell, 2007, Matten and Moon, 2008) shows that CSR is a social construct formed out of society's view on the relationship between the corporate entity and the environment within which it operates.

Following from this, CSR which originated in the global north takes its current definition and construction of the concept of CSR is steeped in the concerns of investors, companies, campaign groups and consumers based in the world's wealthiest countries, with little or no input from middle to low-income countries (Ward, Wilson and Zarsky, 2007; Frederick, 2006; Cavrou, 1999). Therefore CSR practices framed in these rich countries have been internationalised through globalisation, aid and international trade and investment, to other social contexts.

As the concept has evolved and spread globally, there have been calls for context-based studies because much of what we know about CSR through research is from the developed economic context (Dobers and Halme, 2009). These calls have been profound for developing countries because of their unique business systems and institutional structures, which may affect how CSR is expressed (Jamali and Neville 2011; Okoye 2012a). Dobers and Halme (2009) argue that the need for CSR in developing countries is more pronounced because of social legislation that is apparently less comprehensive and poorly enforceable (Cheruiyot and Osando, 2016) that result in shortfalls in the provision of social goods. Further, Julian and Ofori-Dankwa (2013) believe the expression of CSR in developing countries may be different from that of developed countries due to institutional differences that affect the firms' access to the resources needed to undertake the CSR oriented activities.
For one, the idiosyncratic way in which businesses and national institutions configure their operations means that CSR is expressed in different ways in different settings (Jamali and Neville 2011; Okoye 2012a). Thus Western standards of CSR may not be directly applicable in the global south. Further, the developmental challenges that bemuse such regions influence the expectations of the public on CSR (Nadaf and Nadaf, 2014). Dartey-Baah and Amponsah-Tawiah (2011)~argue that public expectations in the weak, corrupt or under-resourced governments contexts that characterize developing countries, is often towards affirmative action and filling in the governance gaps that exist
According to Wright et al.~(2005, p.~7) ``firms competing within emerging economies face a `high velocity' environment of rapid political, economic, and institutional changes that are accompanied by relatively underdeveloped factor and product markets''. This changing environment presents different challenges for firms operating in these countries which have been widely documented in the literature (for example Fornes and Butt-Philip, 2009; Guillen, 2000; Hoskisson et al., 2000; Khanna and Palepu, 1997; Khanna and Palepu, 2000; Peng, 2003). To this, Wright et al.~(2005) added that emerging markets are ``a new context in which to understand the relative strengths and weaknesses of the different conceptual perspectives'' used in conventional theory (p.~2).

One of the most well-known discussions on CSR in developing countries is that of Visser (2008). In his paper, he emphasised the need to explore CSR specific to the developing country context because of the unique characteristics of developing countries that make existing knowledge of CSR which was mostly gained in developed countries not directly transferable. According to him,
``The rationale for focusing on CSR in developing countries as distinct from CSR in the developed world is fourfold:
developing countries represent the most rapidly expanding economies, and hence the most lucrative growth markets for business (IMF, 2006);
developing countries are where the social and environmental crises are usually most acutely felt in the world (WRI, 2005; UNDP, 2006);
developing countries are where globalization, economic growth, investment, and business activity are likely to have the most dramatic social and environmental impacts (both positive and negative) (World Bank, 2006); and
developing countries present a distinctive set of CSR agenda challenges which are collectively quite different to those faced in the developed world. ``(Visser 2008, pg 2)

CSR in developing countries is shaped by culture and traditions, political reforms, Socio-economic priorities, governance gaps, crises response, market access, international standardization, investment incentives, stakeholder activism and supply chain. Visser (2006), concludes by developing a context-specific CSR pyramid for developing countries which shows that, unlike that of Carrol (1991), CSR in developing countries emphasises the economic and philanthropic aspects more that the legal and ethical issues. CSR is closely linked with the principle of sustainable development, which argues that enterprises should make decisions based not only on financial factors such as profits or dividends but also based on immediate and long term social and environmental consequences of its activities.

\begin{verbatim}
        2.2.2 ORGANISATIONAL LEVEL DETERMINANTS
\end{verbatim}

Organisational factors refer to the structural, governance and performance characteristics of firms that affect the way firms define their stakeholders and engage with them. For our purposes, we focus on the effect of financial performance on CSP

\begin{verbatim}
            2.2.2.1 THE EFFECT OF CORPORATE FINANCIAL PERFORMANCE ON CORPORATE SOCIAL PERFORMANCE
\end{verbatim}

In a market economy, financial incentives are likely to be the most significant determinant of company behaviour; therefore, financial performance can be expected to affect the level of corporate social performance. Carroll (1979, 1991) conceptualized a hierarchy of corporate social responsibilities as economic, legal, ethical and discretionary/philanthropic. Drawing from this, firms can only engage in CSR after its economic obligations to shareholders have been fulfilled. After legal responsibilities and ethical responsibilities, last in the hierarchy are philanthropic responsibilities (Carroll, 1991), labelled initially as discretionary responsibilities (Carroll, 1979). Thus, especially for in the case of philanthropy, which has been determined to be at the discretion of management (Aupperle,Carroll,andHatfield, 1985; O'Neill, Saunders,and McCarthy, 1989 Maignan, 2001) the availability of discretionary resources has been found to be critical (Ullmann, 1985). If corporate social responsibility is viewed as a high cost, firms with relatively high past financial performance may be more willing to absorb these costs in the future (Parket and Eibert, 1975; Ullmann, 1985). In contrast, less profitable firms may be less willing to undertake socially responsible actions. (McGuire and Spurgeon, 1988). This school of thought is backed by the slack resources theory.

\hypertarget{slack-resources-theory}{%
\paragraph{SLACK RESOURCES THEORY}\label{slack-resources-theory}}

Slack resources theory has been the primary theoretical grounding of CSR antecedent research and has directed attention toward the effects of financial resource availability on CSR expenditures (e.g.~McGuire, Sundgren, and Schneeweis, 1988; Ullmann, 1985). Slack resources are spare or uncommitted resources, a cushion of resources beyond the minimum necessary to maintain the organizational coalition (Cyert and March, 1963), or excess resources beyond those needed to produce a given level of output (Nohria and Gulati, 1996). The term has been defined as ``potentially usable resources'', its literal meaning, (George, 2005) or a conceptual one, related to prior financial performance or profitability (Margolis and Walsh, 2003; Preston and O'Bannon, 1997). In terms of empirical research, slack-resources have been operationalized for measuring its literal meaning, as debt to equity (Alessandri, 2008) or current assets to current liabilities (Bansal, 2005; Strike et al., 2006). In CSP research, however, the usage of slack-resources is more related to its conceptual notion (Hillman and Keim, 2001; Orlitzky et al., 2003). In this sense, slack has been operationalised as a measure of financial performance. Essentially, organizational slack, under this context, is a consequence of doing well (Seifert et al., 2004). Thus, from an initially low baseline of CSR when financial resource availability is meagre, the firm's ability and propensity to engage in social involvement rises along with financial resource availability (Preston and O'Bannon, 1997; Seifert et al., 2004). A good financial performance implies profitability, which leads to available funds (Preston and O'Bannon, 1997; Waddock and Graves, 1997). Slack-resources provide firms with the possibility of investing in initiatives that do not seem an immediate pay-off (Bansal, 2005), or that are not exactly a priority.

The importance of slack resources theory for CSR expenditures have been clarified and confirmed in literature (Adams and Hardwick, 1998; Brammer and Millington, 2004; Preston and O'Bannon, 1997; Saiia, Carroll, and Buchholtz, 2003; Seifert, Morris, and Bartkus, 2004; Waddock and Graves, 1997(e.g.~Aguilera-Caracuel et al., 2015; Brammer and Pavelin, 2006; Julian and Ofori-Dankwa, 2013; Leyva-de la Hiz et al., 2018; Waddock and Graves, 1997).), and more recent researchers still investigate its posited relationships and employ the theory's logic (Amato and Amato, 2007; Brammer and Millington, 2008; Surroca et al., 2010). More recently, researchers grouped slack resources into absorbed and unabsorbed according to the degree of their absorption into ongoing activities (Sharfman et al., 1988; Suzuki, 2018; Tan and Peng, 2003; Voss et al.~2008).

Unabsorbed slack refers to excess resources uncommitted to particular organisational activities, resources that can be re-deployed for other purposes. Conversely, absorbed slack refers to excess resources tied to specific operations that cannot be re-deployed for other purposes (Tan and Peng, 2003; Voss et al., 2008). Examples of unabsorbed slack are excess financial liquid assets i.e.~cash or credit lines (Voss et al., 2008) and ``employees' redundant time'' (Bowen, 2002 p.~306), whereas examples of absorbed slack are dedicated investments such as excess production capacity (Voss et al., 2008), specialized skilled labour (Greve, 2003a), and excess inventory levels (Bourgeois, 1981). Following from existing literature such as Hillman and Keim, 2001 Orlitzky et al., 2003, we operationalize slack as financial performance. The next section discusses the measurement of financial performance for the purposes of corporate social performance research.

\begin{verbatim}
            2.2.2.3 MEASURES OF CORPORATE FINANCIAL PERFORMANCE
\end{verbatim}

Previous operationalisations of corporate financial performance (CFP) in existing corporate social performance literature can be grouped into three broad categories: market-based (investor returns), accounting-based (accounting returns), and perceptual (survey) measures.

Firstly, based on the assumption that shareholders are the primary stakeholder group whose satisfaction determines the company's fate (Cochran and Wood 1984), market-based measures are used as proxies for firm financial performance. The bidding and asking processes of stock-market participants, who rely on their perceptions of past, current, and future stock returns and risk, determine a firm's stock price and thus market value (Galant and Cadez, 2017). Therefore, changes in the stock price are a reflection of the performance of the firm as measured by the market participants. The chief advantage of market-based measures is that it is contemporaneous. This means that they reflect changes in CSR faster than accounting-based measures. The most significant limitation of market-based measures is that they are only available for publicly listed companies. In addition, market-based measures inevitably incorporate systematic (not-firm-specific) market characteristics (e.g., recession), (McGuire et al., 1988).

Alternatively, accounting-based indicators, such as the firm's return on assets (ROA), return on equity (ROE), or earnings per share (EPS), capture a firm's (Cochran and Wood 1984) as they a product of internal decision-making capabilities and managerial performance through the allocation of discretionary funds to different projects and policy (such as CSP oriented activities). Accounting-based measures have the advantage of being available for all companies and being standardized in a way that allows for easy comparison. However, they have the handicap of being historical. Further, while total categories (e.g., net profit) fail to take company size into account (Al-Tuwaijri et al., 2004), relativised accounting ratios like return on assets (ROA) may be biased if the sample includes companies from different industries (due to the varying age and structure of assets across industries) (Galant and Cadez, 2017).

In an attempt to combine the strengths and reduce the weaknesses of the two measures, some studies combine the two measures by using indicators such as Tobin's Q (market value/total assets) or MVA (market value--book value of equity and debt) (Garcia-Castro, Ariño, and Canela, 2010; Rodgers, Choy, and Guiral, 2013). Others have also tried to derive a comprehensive measure of financial performance by combining different existing measures to form one integrated index (see, for example, Peng and Yang, 2014). Similarly, the Zmijewski score (measured using a-- a construct based on a company's profitability, liquidity and leverage ratio) has been adopted as a proxy for accounting-based company profitability (Rodgers et al., 2013).

Lastly, perceptual measures of CFP ask survey respondents to provide subjective estimates of, for instance, the firm's `soundness of financial position', `wise use of corporate assets', or `financial goal achievement relative to competitors' (Conine and Madden 1987; Reimann 1975; Wartick 1988). For our purposes, we employ multiple accounting measures of financial performance namely Return on Asset and Return on Equity. This is informed by our focus on the internal decision-making system that determines the level of Non-market Investment (specifically in terms on CSP)

\hypertarget{the-curvilinear-relationship-between-corporate-financial-performance-and-corporate-social-performance}{%
\paragraph{THE CURVILINEAR RELATIONSHIP BETWEEN CORPORATE FINANCIAL PERFORMANCE AND CORPORATE SOCIAL PERFORMANCE}\label{the-curvilinear-relationship-between-corporate-financial-performance-and-corporate-social-performance}}

It is evident from existing literature that the link between Corporate Social Performance and Corporate Financial Performance continues to be controversial and tenuous. Some studies have found a positive relationship between CSR and CFP (Ullmann, 1985; Waddock and Graves, 1997; Van Beurden and Gössling, 2008; Wang and Qian, 2011; Koh, Qian and Wang, 2014) whiles others have found negative results (Brammer, Brooks and Pavelin, 2006). After a long and inconclusive debate on how CSP affects CFP, scholarship has moved beyond the ``allies-and-adversaries'' dichotomy (Ramanathan, 2018) which states that CSP either enhances or weakens CFP perpetually. Recent research endeavours have sort to understand which circumstances improve the relationship and which undermine it (Chen et al., 2018; Maqbool and Zameer, 2018). One perspective which is garnering momentum is the curvilinear hypothesis which posits that the relationship between CSP is not static but dynamic (Ramanathan, 2018). In other words, as CSP increases (or falls) CFP will increase (or fall) till it reaches a maximum (or minimum) and then it will begin to fall (or rise).

Following from microeconomic theory, literature that examines the curvilinear relationship has asserted that it is U shaped (Chen et al., 2018). Thus, as CSP investments begin to rise, firm performance will decline due to the cost of the structural adjustment and agency cost that is required to increase CSP engagement (Wang et al., 2016). However, after some level of investment, CSR will begin to pay off by increasing access to critical resources such as customer goodwill and protecting against losing those resources (for instance through lawsuits that result in loss of reputation). Further, drawing from the organisation's learning curve perspective, McWilliams and Siegel, (2001) posit that explicit and implicit costs of CSR activities may be remedied by companies' learning curve on social engagements. Therefore, after firms have invested in CSR overtime, the marginal cost of the social projects and initiatives they engage in begins to fall as they learn to take synergistic opportunities available. Thus the effect of the CSR investment cost on profitability lowers and financial performance increases.

In a bid to test these hypotheses, Barnett and Salmon (2006) examined how the financial performance of mutual funds relates to the number of social screens used in social investment funds. They found that as the number of social screens used in investment appraisal increased, the financial returns declined at first but then rebound as the screens reach a maximum. Similarly, Park and Lee's (2009) examination of the relationship between reputational ratings and accounting-based CFP found a U shaped effect. Again, Wang et al.~(2016) found that in the international construction industry, the relationship between CSR and CFP (as measured by return on assets and earnings per share) is curved. Han et al.(2016) expanded the definition of social responsibility to include governance activities. Their study confirmed a negative (or U shaped) relationship between environmental activity and CFP, whereas the governance responsibility performance score presented a positive (or inverse U shaped) relationship.

In another instance, however, Wang et al.~(2008) found that as corporate philanthropy (amount spent on charity) increases, Corporate Financial Performance (measured by return on assets and Tobin's Q) also increases to a maximum point and then begins to fall. Thus a curved relationship exists between the two variables though inverted. Furthermore, Lankoski's (2008) model demonstrates an inverted-U relationship between CSR outcomes and economic performance, such that as the marginal costs of CSR activities increase, the marginal revenues decline, reach a minimum and then increase.

Overall, empirical evidence shows that the intensity of the influence of CSR costs and benefits on firm performance constantly changes; thus a linear model is insufficient in explaining the relationship (Salzmann et al 2005). Drawing from the preceding arguments, we hypothesis the U shaped relationship is also true for the effect of corporate financial performance on corporate social responsibility.

One perspective sees financial performance as a determinant of social responsibility whiles another sees it as a consequence. The current study seeks to determine whether this relationship is curvilinear. Drawing from the perspective that CSP is discretionary (at least in part) (Carroll, 1979) the decision to invest in it will be affected by the availability of excess resources. Therefore, we posit that as financial performance increases, firms' access to slack resources increase and so does their capacity to invest more into Corporate Social Performance to a maximum and then it will decline. However, as the utility derived from CSR investments peak, further investments in CSR will decrease in spite of rising financial performance. Therefore we hypothesize that there is an inverted U-shaped relationship between financial performance and corporate social performance.

\hypertarget{contingencies-of-the-corporate-financial-performance-corporate-social-responsibility-relationship}{%
\paragraph{CONTINGENCIES OF THE CORPORATE FINANCIAL PERFORMANCE -- CORPORATE SOCIAL RESPONSIBILITY RELATIONSHIP}\label{contingencies-of-the-corporate-financial-performance-corporate-social-responsibility-relationship}}

The contingency theory of CSP as propounded by Husted (2000) suggests that CSP is not universal but contingent on the nature of the social issue. The contingency perspective hypothesizes that CSP is the outcome of the fit between CSP's endogenous organizational variables and exogenous contextual variables (Husted, 2000; Rowley and Berman, 2000; McWilliams and Siegel, 2001). In other words, organizational performance depends on contextual variables (Gerdin and Greve, 2004) and the fit between the organization's socially responsible practices and the contextual variable determines the attainment of organizational performance (Doty et al., 1993; Gerdin and Greve, 2004).

The contingency theory has been found to be critical to CSP research, and its neglect is cited as contributing to the inconsistencies in results of CSP studies (Ullmann, 1985; Waddock and Graves, 1997; McWilliams and Siegel, 2001; Van Beurden and Gössling, 2008). The moderation approach of contingency theory assumed that the impact of an independent
variable on the dependent variable is contingent on the level of a third variable, the so-called moderator (Gerdin and Greve, 2004). That is, the underlying theory specifies that the third variable moderates the effect that the independent variable has on the dependent variable (Luft and Shields, 2006). A plethora of studies has investigated various moderators of the CSP-CFP relationship such as firms' size, age, ownership, strategic orientation, innovation, and differentiation, and managerial persona. This study examines the moderating effect of corporate governance factors (specifically board characteristics) on how Corporate Financial Performance determines Corporate Social Performance.

\hypertarget{board-characteristics}{%
\paragraph{BOARD CHARACTERISTICS}\label{board-characteristics}}

The characteristics of a firms board of directors is a critical determinant of CSP because of the role they play in guiding a firm's strategic orientations and decision making (Schulze et al., 2001; Walsh and Seward, 1990). Research has thus examined how characteristics such as board independence, orientation, diversity, and size affect CSR activities.

The size of the board, measured as the number of directors on the board affects the quality of board decisions (Dalton et al., 1999). A larger board offers a larger array of expertise to draw on for improved decision making (Zahra and Pearce, 1989). Further, the neo-institutional logic and stakeholder theory predicts that large boards are representative of diverse interests (Hillman and Keim, 2001; Kock et al., 2012) and can help garner CSR investments. As per the Resource-Based Theory, larger boards imply better social capital (Pfeffer and Salancik, 1978) and balanced decision making that can result. However, Goodstein et al.~(1994) suggest that large boards are not as apt to initiate strategic action because they are less participative and cohesive than smaller ones. Further, the Agency Theory contends that large-sized boards often face free-rider problems (Dalton et al., 1998) as well as coordination and communication issues (Jensen, 1993). In this scenario, there is a likelihood of boards being dominated by short-term profit-oriented managers who can steer firms to reduce CSR investments (Walls and Hoffman, 2013).

Studies on board independence show mixed results. For instance, non-executive directors are positively associated with the people and product aspects of corporate social performance (Johnson and Greening, 1999), socially responsible behaviour of firms (Webb, 2004), and discretionary dimensions of CSR (Ibrahim, Howard and Angelidis, 2003), and negatively related to environmental litigations (Kassinis and Vafeas, 2002). However, other studies find no relationship between non-executive directors and corporate philanthropy (Brown et al., 2006; Coffey and Wang, 1998), legal and ethical dimensions of CSR (Ibrahim et al., 2003), and environmental violations (McKendall, Sanchez, ´ and Sicilian, 1999). In addition, Wang and Dewhirst (1992) find that non executive directors view some stakeholder groups differently than executive directors do.

Another aspect of board structure often investigated for corporate social performance is diversity. Specifically, many studies have considered the impact of the proportion of female directors on boards. Most of these studies identify a positive association between proportion of women directors and socially responsible firms (Webb, 2004) and Corporate Social Performance (Coffey and Fryxell, 1991; Stanwick and Stanwick, 1998). Female directors additionally have a positive impact on firm reputation, after being mediated by CSR (Bear, Rahman, and Post, 2010).

CEO duality occurs when the functional role of the CEO (management) and that of the chairman (control) are vested in the same individual elevating him/her to an entrenched position within the firm (Rechner and Dalton, 1991). From an agency perspective, CEO duality leads to a concentration of managerial power (Surrocaand Tribo, 2008), enabling managers to suspend CSR investments, if considered wasteful. In contrast, dual board leadership separates management and control, consequently enhancing boards' monitoring power (Fama and Jensen, 1983). Following Stakeholder Theory, dual board leadership can improve social capital and stakeholder representation within boards to influence CSR (Finkelstein and Hambrick, 1990) positively.

Aside from the monitoring role of the board propagated by Agency Theory, Resource-Based Theory asserts that board interlocking and breadth of director experiences enhance the human and social capital of directors and influence the nature and quality of managerial--board interactions (Westphal, 1999), thereby stimulating potential adoption of CSR practices (Shropshire, 2010). In line with RDT, our review identifies mostly positive effects between board interlocks, board experience, and CSR (Glass, Cook, and Ingersoll, 2015; Kassinis and Vafeas, 2002;Walls and Hoffman, 2013) with some neutral effects (De Villiers et al., 2011). Although social networks of directors have been extensively explored in the Corporate Governance literature in relation to financial performance, this research is nascent in the context of as CSR (Galant and Cadez, 2017). Based on the above, we expect that board characteristics will moderate the relationship between Corporate Financial Performance and Corporate Social performance.
.
2.3 CONCEPTUAL FRAMEWORK AND HYPOTHESIS DEVELOPMENT
Following from the review of literature, the study seeks to examine the determinants of Non-Market Investment (measured as Corporate Social Performance) in Africa.

First and foremost, drawing from the Institutional Theory which establishes that social structures, exert significant influence on the corporations' decision-making (e.g., Campbell, 2007; Campbell, Hollingsworth, and Lindberg, 1991) and the findings of Visser (2009) that the content of CSP in developing countries is at variance with mainstream CSR (Carroll's (1991) CSR pyramid), we explore what constitutes CSR in the Sub-Sahara African region. The first phase of the study, therefore, examines which dimensions of CSR are reported on by listed firms in Sub-Saharan Africa. We adapt the dimensions of CSP Fatma, Rahman and Khan's (2014) multi-stakeholder measure of CSP. We employ the Principal Component Factor analysis method in determining which dimensions of CSP are used within the African context.

Secondly, based on the slack resources theory, we explore the effect of Corporate Financial Performance on Corporate Social Performance and test for the non-linearity of the relationship. We employ accounting based measures of CFP to reflect the internal efficiencies of the firm. Thus we measure Corporate Financial performance as Return on Asset (ROA) and Return on Equity (ROE).

We hypothesis that:
Hypothesis 1: ROA has an inverted U-shaped relationship with CSP
Hypothesis 2: ROE has an inverted U-shaped relationship with CSP

Again, we posit that due to the strategic role played by the board of directors in guiding a firm's orientation and decision making (Schulze et al., 2001; Walsh and Seward, 1990), they determine the extent to which slack resources will be allocated to CSR activities. Following from our perusal of existing literature, we posit that:

Hypothesis 3: Board Gender Diversity moderates the curvilinear relationship between ROA and CSP
Hypothesis 4: Board Gender Diversity moderates the curvilinear relationship between ROE and CSP

Figure 2 1 CONCEPTUAL FRAMEWORK FOR STUDY

Researchers Construct, (2020)

\end{document}
